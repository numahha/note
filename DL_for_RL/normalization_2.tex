\section{計算途中:正規化 (normalization)}
\subsection{batch normalization \cite{ioffe2015batch}}
多層NNのパラメータを更新する場合,前の層のパラメータの変化によって次の層に対する入力の分布が大きく変化してしまい,学習が不安定になりやすい.
正規化のアイデアは,入力を標準正規(ガウス)分布を使って変換して用いることにより,学習を安定化させようというものである.

$n$番目のデータの$\ell$層目の中間変数を,$\vz^{(\ell)}_n=(z^{(\ell)}_{n,1},z^{(\ell)}_{n,2},\cdots)^T$と表記する.
ミニバッチに含まれるデータのインデックスの集合を$\mathcal{S}$とする.
具体的には,次の変換を用いる.
\begin{align*}
\mu_{d}^{(\ell)} &= \frac{1}{m}\sum_{n\in\mathcal{S}}z^{(\ell)}_{n,d}
\\
{\sigma^2}^{(\ell)}_{d}
&=
\frac{1}{m} \sum_{n\in\mathcal{S}} (z^{(\ell)}_{n,d}-\mu_{d}^{(\ell)})^2
\\
\hat{z}^{(\ell)}_{n,d}
&=
\frac{z^{(\ell)}_{n,d}-\mu_{d}}
{\sqrt{{\sigma^2}^{(\ell)}_{d}+\epsilon_c}}
\\
\bar{z}^{(\ell)}_{n,d}
&=
\gamma^{(\ell)}_d \hat{z}^{(\ell)}_{n,d} + \beta^{(\ell)}_d
\end{align*}
ここで,$\gamma_d$と$\beta_d$は学習可能な係数である.

この操作を,次のように略記する.
\begin{align*}
\vmu^{(\ell)}, {\vsigma^2}^{(\ell)} &= \mbox{NormParam}(\{\vz^{(\ell)}_{n}\}_{n\in\mathcal{S}})
\\
\bar{\vz}^{(\ell)}_{n}
&=
\mbox{BN}(\vz^{(\ell)}_n,\vmu^{(\ell)}, {\vsigma^2}^{(\ell)},\vgamma^{(\ell)},\vbeta^{(\ell)})
\end{align*}

\paragraph{MLP順伝播 with バッチ正則化.}
各層においてバッチの平均と分散をとる操作があるので,バッチ単位で層毎に順伝播を行う.
\begin{align*}
 \vz^{(\ell)}_n &=
 \left\{
     \begin{array}{ll}
       \vx_n & \mbox{if }\ell=0 \\
       \vphi.(\va^{(\ell)}_n) & \mbox{if }\ell \in \{1,\cdots,L-1 \}
     \end{array}
   \right.
   ,~~~\mbox{for }n\in\mathcal{S}
   \\
\vmu^{(\ell)}, {\vsigma^2}^{(\ell)} &= \mbox{NormParam}(\{\vz^{(\ell)}_{n}\}_{n\in\mathcal{S}}),~~~\ell=0,\cdots,L-1
\\
\bar{\vz}^{(\ell)}_{n}
&=
\mbox{BN}(\vz^{(\ell)}_n,\vmu^{(\ell)}, {\vsigma^2}^{(\ell)},\vgamma^{(\ell)},\vbeta^{(\ell)}),~~~\mbox{for }n\in\mathcal{S},~~~\ell=0,\cdots,L-1
\\
 \va^{(\ell)}_n &= \vW^{(\ell)}\bar{\vz}^{(\ell-1)}_n,~~~\mbox{for } n\in\mathcal{S},~~~\ell=1,\cdots,L
 \\
 \vy_n &= \va^{(L)}_n + \vepsilon_n,~~~\mbox{for } n\in\mathcal{S}
\end{align*}


次に,2層逆伝播を導出する.ミニバッチに対する誤差関数を$E(\vW,\vgamma,\vbeta)= \sum_{n\in\mathcal{S}}E_n(\vW,\vgamma,\vbeta)$とすると,
\begin{align*}
 \frac{\partial E}{\partial w^{(2)}_{d,h_1}}
 &=
 \sum_{n\in\mathcal{S}}\frac{\partial E_n}{\partial a^{(2)}_{n,d}} \frac{\partial a^{(2)}_{n,d}}{\partial w^{(2)}_{d,h_1}}
 =
 \sum_{n\in\mathcal{S}}\frac{\partial E_n}{\partial a^{(2)}_{n,d}} \bar{z}_{n,d}^{(2)}
 =
 \sum_{n\in\mathcal{S}} \delta^{(2)}_{n,d} \bar{z}_{n,d}^{(2)}
 \\
 \frac{\partial E}{\partial w^{(1)}_{h_1,h_0}}
 &=
 \sum_{n\in\mathcal{S}}
 \left\{
 \left(
 \sum_{d=1}^{D}
 \frac{\partial E_n}{\partial a^{(2)}_{n,d}}
 \frac{\partial a^{(2)}_{n,d}} {\partial \bar{z}^{(1)}_{n,h_1}}
 \right)
 \frac{\partial \bar{z}^{(1)}_{n,h_1}}{\partial w^{(1)}_{h_1,h_0}}
 \right\}
 \\
 &=
 \sum_{n\in\mathcal{S}}
 \left\{
 \left(
 \sum_{d=1}^{D}
 \delta^{(2)}_{n,d}
 \frac{\partial} {\partial \bar{z}^{(1)}_{n,h_1}}
 \left(
 \sum_{h_1'=1}^{H_1}w^{(2)}_{d,h_1'}\bar{z}^{(1)}_{n,h_1'}
 \right)
 \right)
 \frac{\partial \bar{z}^{(1)}_{n,h_1}}{\partial w^{(1)}_{h_1,h_0}}
 \right\}
 \\
 &=
 \sum_{n\in\mathcal{S}}
 \left\{
 \left(
 \sum_{d=1}^{D}
 \delta^{(2)}_{n,d}
 w^{(2)}_{d,h_1}
 \right)
 \frac{\partial \bar{z}^{(1)}_{n,h_1}}{\partial w^{(1)}_{h_1,h_0}}
 \right\}
\end{align*}
各項を計算していくと,
\begin{align*}
 \frac{\partial \bar{z}^{(1)}_{n,h_1}}{\partial w^{(1)}_{h_1,h_0}}
 &=
 \frac{\partial}{\partial w^{(1)}_{h_1,h_0}} (\gamma_{h_1} \hat{z}^{(1)}_{n,h_1} + \beta_{h_1})
 =
 \gamma_{h_1} \frac{\partial}{\partial w^{(1)}_{h_1,h_0}}
  \left( \frac{z^{(1)}_{n,h_1}-\mu_{h_1}}
 {\sqrt{{\sigma^2}^{(1)}_{d}+\epsilon_c}} \right)
 \\
 &=
 \gamma_{h_1} \sum_{h_1'}\frac{\partial z^{(1)}_{n,h_1'}}{\partial w^{(1)}_{h_1,h_0}}
  \frac{\partial }{\partial z^{(1)}_{n,h_1'}}
  \left( \frac{z^{(1)}_{n,h_1}-\mu_{h_1}}
 {\sqrt{{\sigma^2}^{(1)}_{h_1}+\epsilon_c}} \right)
\end{align*}
ここで,$\vz^{(1)}=\vphi.(\va^{(1)})$が要素ごとの積であることに注意すれば,
\begin{align*}
\frac{\partial z^{(1)}_{n,h_1'}}{\partial w^{(1)}_{h_1,h_0}}
=
\frac{\partial z^{(1)}_{n,h_1'}}{\partial a^{(1)}_{n,h_1'}}
\frac{\partial a^{(1)}_{n,h_1'}}{\partial w^{(1)}_{h_1,h_0}}
=
\phi'(a^{(1)}_{n,h_1'})
\frac{\partial}{\partial w^{(1)}_{h_1,h_0}}\sum_{h_0'} w^{(1)}_{h_1',h_0'}z^{(0)}_{n,h_0'}
=\left\{
    \begin{array}{ll}
      \phi'(a^{(1)}_{n,h_1})z^{(0)}_{n,h_0} & \mbox{if }h_1=h_1' \\
      0 & \mbox{if } h_1\neq h_1'
    \end{array}
  \right.
\end{align*}
これを代入すると,
\begin{align*}
 \frac{\partial \bar{z}^{(1)}_{n,h_1}}{\partial w^{(1)}_{h_1,h_0}}
 =
 \gamma_{h_1} \phi'(a^{(1)}_{n,h_1})z^{(0)}_{n,h_0}
  \frac{\partial }{\partial z^{(1)}_{n,h_1}}
  \left( \frac{z^{(1)}_{n,h_1}-\mu_{h_1}}
 {\sqrt{{\sigma^2}^{(1)}_{h_1}+\epsilon_c}} \right)
\end{align*}
残った偏微分の項を計算すると,
\begin{align*}
  \frac{\partial }{\partial z^{(1)}_{n,h_1}}
  \left( \frac{z^{(1)}_{n,h_1}-\mu_{h_1}}
 {\sqrt{{\sigma^2}^{(1)}_{h_1}+\epsilon_c}} \right)
 &=
 \frac{1}
{\sqrt{{\sigma^2}^{(1)}_{h_1}+\epsilon_c}}
\left\{
 \frac{\partial }{\partial z^{(1)}_{n,h_1}}(z^{(1)}_{n,h_1}-\mu_{h_1})
 -
 \frac{(z^{(1)}_{n,h_1}-\mu_{h_1})}{2({\sigma^2}^{(1)}_{h_1}+\epsilon_c)}\frac{\partial }{\partial z^{(1)}_{n,h_1}}({\sigma^2}^{(1)}_{h_1}+\epsilon_c)
 \right\}
 \\
 &=
 \frac{1}
{\sqrt{{\sigma^2}^{(1)}_{h_1}+\epsilon_c}}
\left\{
 1-\frac{1}{m}
 -
 \frac{(z^{(1)}_{n,h_1}-\mu_{h_1})}{2({\sigma^2}^{(1)}_{h_1}+\epsilon_c)}\frac{2(z^{(1)}_{n,h_1}-\mu_{h_1})(1-\frac{1}{m})}{m}
 \right\}
 \\
 &=
 \frac{1}
{\sqrt{{\sigma^2}^{(1)}_{h_1}+\epsilon_c}}
\left(1-\frac{1}{m}\right)
\left\{
 1
 -
 \frac{(z^{(1)}_{n,h_1}-\mu_{h_1})^2}{m({\sigma^2}^{(1)}_{h_1}+\epsilon_c)}
 \right\}
\end{align*}
最終的に,次のように書ける(合ってる?).
\begin{align*}
\frac{\partial E}{\partial w^{(1)}_{h_1,h_0}}
&=
\sum_{n\in\mathcal{S}}
\left\{
\frac{\gamma_{h_1} \phi'(a^{(1)}_{n,h_1})z^{(0)}_{n,h_0}(1-\frac{1}{m})}
{\sqrt{{\sigma^2}^{(1)}_{h_1}+\epsilon_c}}
\left[
1
-
\frac{(z^{(1)}_{n,h_1}-\mu_{h_1})^2}{m({\sigma^2}^{(1)}_{h_1}+\epsilon_c)}
\right]
\left(
\sum_{d=1}^{D}
\delta^{(2)}_{n,d}
w^{(2)}_{d,h_1}
\right)
\right\}
\end{align*}

●●●TODO●●●バッチ正則化の誤差逆伝播の導出と検算法.

\paragraph{論文\cite{ioffe2015batch}の導出.}
計算チェックのために書く.
\begin{align*}
 \frac{\partial l}{\partial \hat{x}_i}
 &=
 \gamma\frac{\partial l}{\partial y_i}
 \\
 \frac{\partial l}{\partial \sigma^2}
 &=
 -\frac{1}{2}(\sigma^2+\epsilon)^{-\frac{3}{2}}\sum_{i=1}^m(x_i-\mu)\frac{\partial l}{\partial \hat{x}_i}
 \\
 &=
 -\frac{\gamma}{2}(\sigma^2+\epsilon)^{-\frac{3}{2}}\sum_{i=1}^m(x_i-\mu)\frac{\partial l}{\partial y_i}
 \\
 \frac{\partial l}{\partial \mu}
 &=
 -(\sigma^2+\epsilon)^{-\frac{1}{2}}\sum_{i=1}^m\frac{\partial l}{\partial \hat{x}_i} +
 \frac{-2\sum_{i=1}^m(x_i-\mu)}{m}\frac{\partial l}{\partial \sigma^2}
 \\
 &=
 -\gamma(\sigma^2+\epsilon)^{-\frac{1}{2}}\sum_{i=1}^m\frac{\partial l}{\partial y_i}
 \\
 \frac{\partial l}{\partial x_i}
 &=
  -(\sigma^2+\epsilon)^{-\frac{1}{2}}\frac{\partial l}{\partial \hat{x}_i}
  +\frac{2(x_i-\mu)}{m}\frac{\partial l}{\partial \sigma^2}
  +\frac{1}{m}\frac{\partial l}{\partial \mu}
 \\
 &=
 \gamma(\sigma^2+\epsilon)^{-\frac{1}{2}}
 \left\{
 \frac{\partial l}{\partial y_i}
 -(\sigma^2+\epsilon)^{-1}\frac{2(x_i-\mu)}{m}\sum_{j=1}^m(x_j-\mu)\frac{\partial l}{\partial y_j}
 -\frac{1}{m}\sum_{j=1}^m\frac{\partial l}{\partial y_j}
 \right\}
 \\
 &=
 \gamma(\sigma^2+\epsilon)^{-\frac{1}{2}}
 \left\{
 \frac{\partial l}{\partial y_i}
 -\frac{2(x_i-\mu)}{m(\sigma^2+\epsilon)}\sum_{j=1}^m(x_j-\mu)\frac{\partial l}{\partial y_j}
 -\frac{1}{m}\sum_{j=1}^m\frac{\partial l}{\partial y_j}
 \right\}
\end{align*}
