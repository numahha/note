\section{再パラメータ化}
\label{sec:reparameterization}
変分ベイズをやる時に,解析的に得られない項がありサンプル近似をする必要がある.
雑に言うと,再パラメータ化は,変数変換を用いてサンプル近似の性能を向上する方法である.


\subsection{復習:変分ベイズ}
入出力データを$(\vX,\vY)$,パラメータを$\vw$とする.
確率モデル$p(\vY|\vw,\vX)$とする.
事前分布$p(\vw)$として,Bayes則に従って事後確率を次のように得る.
\begin{align*}
p(\vw|\vY,\vX)
=
\frac{P(\vY|\vw,\vX)P(\vw)}{P(\vY|\vX)}
=
\frac{P(\vY,\vw|\vX)}{P(\vY|\vX)}
\end{align*}
一般に,$p(\vw|\vY,\vX)$は解析的な形にはならない.
変分ベイズでは,$q(\vw;\vxi)$を用いて$p(\vw|\vY,\vX)$を近似することを考える($\vxi$は変分パラメータと呼ばれる).
近似の評価指標をKLダイバージェンスとすると,
\begin{align*}
\mbox{KL}\left[
q(\vw;\vxi)||p(\vw|\vY,\vX)
\right]
&=
\int
q(\vw;\vxi)
\frac{\ln q(\vw;\vxi)}{\ln p(\vw|\vY,\vX)} d\vw
=
\int
q(\vw;\vxi)
\left[
\ln q(\vw;\vxi) - \ln p(\vw|\vY,\vX)
\right]
d\vw
\\
&=
\int
q(\vw;\vxi)
\left[
\ln q(\vw;\vxi) - \ln p(\vY,\vw|\vX) + \ln p(\vY|\vX)
\right]
d\vw
\\
&=
\ln p(\vY|\vX)
-
\int
q(\vw;\vxi)
\left[
\ln p(\vY,\vw|\vX) - \ln q(\vw;\vxi)
\right]
d\vw
\\
&=
\ln p(\vY|\vX)
-
{\mathcal{L}}(\vxi)
\end{align*}
ここで,${\mathcal{L}}(\vxi)$は,変分下界(ELBO)として知られる量である.
$\ln p(\vY|\vX)$は$\vxi$に依存しないことに注意すれば,
近似指標$\mbox{KL}\left[q(\vw;\vxi)||p(\vw|\vY,\vX)\right]$を最小化する問題は,
ELBO ${\mathcal{L}}(\vxi)$を最大化する問題と等価である.
ELBOは,次のように変形できる.
\begin{align*}
{\mathcal{L}}(\vxi)
&=
\int
q(\vw;\vxi)
\left[
\ln p(\vY,\vw|\vX) - \ln q(\vw;\vxi)
\right]
d\vw
\\
&=
\int
q(\vw;\vxi)
\left[
\ln p(\vY|\vw,\vX) + \ln p(\vw) - \ln q(\vw;\vxi)
\right]
d\vw
\\
&=
\int
q(\vw;\vxi)
\left[
\ln p(\vY|\vw,\vX)
\right]
d\vw
-
\int
q(\vw;\vxi)
\left[
\ln q(\vw;\vxi) -  \ln p(\vw)
\right]
d\vw
\\
&=
\int
q(\vw;\vxi)
\left[
\ln p(\vY|\vw,\vX)
\right]
d\vw
-
\mbox{KL}\left[
q(\vw;\vxi)||p(\vw)
\right]
\\
&=
\mathbb{E}_{\vw\sim q(\vw;\vxi)}
\left[
\ln p(\vY|\vw,\vX)
\right]
-
\mbox{KL}\left[
q(\vw;\vxi)||p(\vw)
\right]
\end{align*}
基本的には,ELBOの最大化問題を勾配法で解いていくことになるので,上式の各項の$\vxi$に関する微分が欲しいものである.
ここで,$\mbox{KL}\left[q(\vw;\vxi)||p(\vw)\right]$の微分は,事前分布$p(\vw)$と変分分布$q(\vw;\vxi)$を$\vw$の正規分布などで与えれば,解析的に計算することができる.
一方で,$\mathbb{E}_{\vw\sim q(\vw;\vxi)}\left[ \ln p(\vY|\vw,\vX)\right]$の微分は解析的には計算できない.

これを扱うシンプルな方法は,分布$q(\vw;\vxi)$から$\vw$のサンプルを生成してサンプル近似することであるが,実用上は高い分散を生じてしまうことが知られている.


\subsection{reparameterization trick \cite{diederik2014auto}}
与えられた$q(\vw;\vxi)$に対して,
$q(\vg(\vxi;\vepsilon);\vxi)d\vxi=p(\vepsilon)d\vepsilon$を満たす変数変換$\vw=\vg(\vxi;\vepsilon)$と分布$p(\vepsilon)$を定義できる場合には,
次が成り立つ(途中の$(\cdot)_{i,j}$は行列の成分表示,$(\cdot)_{j}$はベクトルの成分表示を指す).
\begin{align*}
\frac{\partial}{\partial \vxi}
\mathbb{E}_{\vw\sim q(\vw;\vxi)}
\left[
\ln p(\vY|\vw,\vX)
\right]
&=
\frac{\partial}{\partial \vxi}
\mathbb{E}_{\vepsilon\sim p(\vepsilon)}
\left[
\ln p(\vY|\vg(\vxi;\vepsilon),\vX)
\right]
\\
&=
\mathbb{E}_{\vepsilon\sim p(\vepsilon)}
\left[
\frac{\partial}{\partial \vxi}
\ln p(\vY|\vg(\vxi;\vepsilon),\vX)
\right]
\\
&=
\mathbb{E}_{\vepsilon\sim p(\vepsilon)}
\left[
\left(
%\left.
\frac{\partial \vg(\vxi;\vepsilon)}{\partial \vxi}
%\right|_{\vepsilon}
 \right)
\left(
\left.
\frac{\partial}{\partial \vw}
\ln p(\vY|\vw,\vX)
\right|_{\vw=\vg(\vxi;\vepsilon)}
\right)
\right]
\\
&=
\mathbb{E}_{\vepsilon\sim p(\vepsilon)}
\left[
\left(
%\left.
\frac{\partial g_j(\vxi;\vepsilon)}{\partial \xi_i}
%\right|_{\vepsilon}
\right)_{i,j}
\left(
\left. \frac{\partial}{\partial w_j} \ln p(\vY|\vw,\vX) \right|_{\vw=\vg(\vxi;\vepsilon)}
\right)_{j}
\right]
\end{align*}
よって,1回のサンプル値$\tilde{\vepsilon}\sim p(\vepsilon)$によって近似すると,
\begin{align*}
\frac{\partial}{\partial \vxi}
\mathbb{E}_{\vw\sim q(\vw;\vxi)}
\left[
\ln p(\vY|\vw,\vX)
\right]
&\approx
\frac{\partial}{\partial \vxi}
\ln p(\vY|\vg(\vxi;\tilde{\vepsilon}),\vX)
=
\left(
%\left.
\frac{\partial g_j(\vxi;\tilde{\vepsilon})}{\partial \xi_i}
%\right|_{\vepsilon}
\right)_{i,j}
\left(
\left. \frac{\partial}{\partial w_j} \ln p(\vY|\vw,\vX) \right|_{\vw=\vg(\vxi;\tilde{\vepsilon})}
\right)_{j}
\end{align*}



\paragraph{具体例:正規分布の場合.}
$q(w;\vxi)$として正規分布が与えられたとする(従って,$\vxi=(\mu_{\xi},\sigma_{\xi})^T$である).
\begin{align*}
 q(w;\vxi)
 =
 \mathcal{N}(w|\mu_{\xi},\sigma^2_{\xi})
 =
 \frac{1}{\sqrt{2\pi\sigma^2_{\xi}}} \exp\left( -\frac{(w-\mu_{\xi})^2}{2\sigma^2_{\xi}} \right)
\end{align*}
ここで,変数変換$w=g(\vxi;\epsilon)$と分布$p(\epsilon)$を次のように定義する.
\begin{align*}
 g(\vxi;\epsilon) &= w = \mu_{\xi}+\sigma_{\xi}\epsilon
 \\
 p(\epsilon) &= {\mathcal{N}}(0,1) = \frac{1}{\sqrt{2\pi}} \exp\left( -\frac{\epsilon^2}{2} \right)
\end{align*}
$dw = \frac{dw}{d\epsilon} d\epsilon = \sigma_{\xi}d\epsilon$であるため,次が成り立つ.
\begin{align*}
 q(w;\vxi)dw
 &=
 \frac{1}{\sqrt{2\pi\sigma^2_{\xi}}} \exp\left( -\frac{((\mu_{\xi}+\sigma_{\xi}\epsilon) -\mu_{\xi})^2}{2\sigma^2_{\xi}} \right) \sigma_{\xi}d\epsilon
 \\
 &=
 \frac{1}{\sqrt{2\pi}} \exp\left( -\frac{\epsilon^2}{2} \right)d\epsilon = p(\epsilon) d\epsilon
\end{align*}
よって,この変数変換と分布は,$q(w;\vxi)dw=p(\epsilon)d\epsilon$を満たしている.
そのため,再パラメータ化が適用できて,
\begin{align*}
\frac{\partial}{\partial \vxi}
\mathbb{E}_{w\sim q(w;\vxi)}
\left[
\ln p(\vY|w,\vX)
\right]
&=
\mathbb{E}_{\epsilon\sim p(\epsilon)}
\left[
\left(
\frac{\partial g(\vxi;\epsilon)}{\partial \xi_i}
\right)_i
\left(
\left. \frac{\partial}{\partial w} \ln p(\vY|w,\vX) \right|_{w=g(\vxi;\epsilon)}
\right)
\right]
\\
&=
\mathbb{E}_{\epsilon\sim p(\epsilon)}
\left[
\left(
     \begin{array}{c}
      \frac{\partial g(\vxi;\epsilon)}{\partial \mu_{\xi} }  \\
      \frac{\partial g(\vxi;\epsilon)}{\partial \sigma_{\xi} }
    \end{array}
\right)
\left(
\left. \frac{\partial}{\partial w} \ln p(\vY|w,\vX) \right|_{w=g(\vxi;\epsilon)}
\right)
\right]
\\
&=
\mathbb{E}_{\epsilon\sim p(\epsilon)}
\left[
\left(
     \begin{array}{c}
      1  \\
      \epsilon
    \end{array}
\right)
\left(
\left. \frac{\partial}{\partial w} \ln p(\vY|w,\vX) \right|_{w=g(\vxi;\epsilon)}
\right)
\right]
\end{align*}

\paragraph{ノート.}
再パラメータ化の拡張や,効率化を与えるメカニズムの解析は,現在でも研究トピックの一つである(例えば\cite{xu2019variance}).
